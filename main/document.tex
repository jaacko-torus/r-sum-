%%%%
%% This is an sample CV template created using altacv.cls
%% (v1.6.3, 04 Oct 2021) written by LianTze Lim (liantze@gmail.com). Now compiles with pdfLaTeX, XeLaTeX and LuaLaTeX.
%%
%% It may be distributed and/or modified under the
%% conditions of the LaTeX Project Public License, either version 1.3
%% of this license or (at your option) any later version.
%% The latest version of this license is in http://www.latex-project.org/lppl.txt
%% and version 1.3 or later is part of all distributions of LaTeX
%% version 2003/12/01 or later.
%%%%

%% Use the "normalphoto" option if you want a normal photo instead of cropped to a circle
% \documentclass[10pt,a4paper,normalphoto]{altacv}

\RequirePackage[hyphens]{url}
\documentclass[10pt,a4paper,ragged2e,withhyper]{altacv}
%% AltaCV uses the fontawesome5 and packages.
%% See http://texdoc.net/pkg/fontawesome5 for full list of symbols.

\usepackage{ragged2e}
\usepackage{xparse}
\usepackage{tikz}
\usepackage{ifthen}
\usepackage{xcolor}

% \usepackage[hyphens]{url}
% \usepackage[hidelinks]{hyperref}

\usepackage{fontspec}
\usepackage[normalem]{ulem}
\urlstyle{same}

\hypersetup{
	colorlinks,
	urlcolor=blue,
}

% Change the page layout if you need to
\geometry{left=1.25cm,right=1.25cm,top=1.5cm,bottom=1.5cm,columnsep=1.2cm}

% The paracol package lets you typeset columns of text in parallel
\usepackage{paracol}

% Change the font if you want to, depending on whether
% you're using pdflatex or xelatex/lualatex
\ifxetexorluatex
	% If using xelatex or lualatex:
	%   \setmainfont{Roboto Slab}
	%   \setsansfont{Lato}
	\setmainfont{Gentium Plus}
	\setsansfont{Gentium Book Basic}

	% \renewcommand{\familydefault}{\sfdefault}
\else
	% If using pdflatex:
	\usepackage[rm]{roboto}
	\usepackage[defaultsans]{lato}
	% \usepackage{sourcesanspro}
	\renewcommand{\familydefault}{\sfdefault}
\fi

% Change the colours if you want to
\definecolor{DarkPastelRed}{HTML}{450808}
\definecolor{GoldenEarth}{HTML}{E7D192}
\definecolor{LightGrey}{HTML}{666666}
\definecolor{PastelRed}{HTML}{8F0D0D}
\definecolor{RoyalBlue}{HTML}{002366}
\definecolor{SlateGrey}{HTML}{2E2E2E}
\definecolor{VeryLightLemon}{HTML}{D5CED0}

\colorlet{accent}{PastelRed}
\colorlet{body}{LightGrey}
\colorlet{cvtag}{LightGrey}
\colorlet{emphasis}{SlateGrey}
\colorlet{hcvtag}{PastelRed}
\colorlet{heading}{DarkPastelRed}
\colorlet{headingrule}{VeryLightLemon}
\colorlet{name}{black}
\colorlet{subheading}{PastelRed}
\colorlet{tagline}{PastelRed}

% Change some fonts, if necessary
\renewcommand{\namefont}{\centering\Huge\rmfamily\scshape}
\renewcommand{\taglinefont}{\centering\large\rmfamily}
\renewcommand{\personalinfofont}{\centering\footnotesize\rmfamily}
\renewcommand{\cvsectionfont}{\LARGE\rmfamily\scshape}
\renewcommand{\cvsubsectionfont}{\large\rmfamily\bfseries}


% Change the bullets for itemize and rating marker
% for \cvskill if you want to
\renewcommand{\itemmarker}{{\small\textbullet}}
\renewcommand{\ratingmarker}{\faCircle}

% "Fork me", @see https://gist.github.com/jnothman/0729018fc39b2c30f082
\usetikzlibrary{shadows.blur}

\newlength{\forkmeoffset}
\setlength{\forkmeoffset}{12em}
\definecolor{forkmebg}{HTML}{CC0000}
\definecolor{forkmefg}{HTML}{FFFFFF}

\NewDocumentCommand{\forkme}{ O{west} O{Fork me on GitHub} m }{
	\ifthenelse{\equal{#1}{east}}{%
		\tikzset{forkmerot/.style={rotate=-45}}
	}{%
		\tikzset{forkmerot/.style={rotate=45}}
	}
	\begin{tikzpicture}[remember picture, overlay]
	\node[forkmerot, shift={(0, -\forkmeoffset)}] at (current page.north #1) {
		\begin{tikzpicture}[remember picture, overlay]
		\node[fill=forkmebg, text centered, minimum width=50em, minimum height=3.0em, blur shadow, shadow yshift=0pt, shadow xshift=0pt, shadow blur radius=.4em, shadow opacity=50, text=forkmefg](fmogh) at (0pt, 0pt) { \href{#3}{#2} };
		\draw[forkmefg!60, dashed, line width=.08em, dash pattern=on .5em off 1.5\pgflinewidth] (-25em,1.2em) rectangle (25em,-1.2em);
		\end{tikzpicture}
	};
	\end{tikzpicture}
}

\ExplSyntaxOn
\NewDocumentCommand{\NewCVTagField}{s m o m}{
	\csdef{CVTag#2}{
		\IfBooleanTF{#1}
			{\cvtag*{#4}}
			{\IfNoValueTF{#3}
				{\cvtag*{#2}[#4]}
				{\cvtag*{#3}[#4]}}
	}
}
\ExplSyntaxOff

% Languages
\NewCVTagField{C}{https://en.wikipedia.org/wiki/C_(programming_language)}
\NewCVTagField{CPlusPlus}[C++]{https://en.wikipedia.org/wiki/C++}
\NewCVTagField{CSharp}[C\#]{https://en.wikipedia.org/wiki/C_Sharp_(programming_language)}
\NewCVTagField{GDScript}{https://en.wikipedia.org/wiki/Godot_(game_engine)\#GDScript}
% \NewCVTagField{Go}{https://en.wikipedia.org/wiki/Go_(programming_language)} % Not enough real world experience yet
\NewCVTagField{Java}{https://en.wikipedia.org/wiki/Java_(programming_language)}
\NewCVTagField{JavaScript}{https://en.wikipedia.org/wiki/JavaScript}
\NewCVTagField{LaTeX}[\LaTeX]{https://en.wikipedia.org/wiki/LaTeX}
\NewCVTagField{Lua}{https://en.wikipedia.org/wiki/Lua_(programming_language)}
\NewCVTagField{Node}{https://en.wikipedia.org/wiki/Node.js}
\NewCVTagField{Python}{https://en.wikipedia.org/wiki/Python_(programming_language)}
\NewCVTagField{Ruby}{https://en.wikipedia.org/wiki/Ruby_(programming_language)}
\NewCVTagField{Scala}{https://en.wikipedia.org/wiki/Scala_(programming_language)}
\NewCVTagField{SQL}{https://en.wikipedia.org/wiki/SQL}
\NewCVTagField{TypeScript}{https://en.wikipedia.org/wiki/TypeScript}
\NewCVTagField*{HTMLCSS}{\href{https://es.wikipedia.org/wiki/HTML}{HTML}/\href{https://en.wikipedia.org/wiki/CSS}{CSS}}
\NewCVTagField*{JSNode}{\href{https://en.wikipedia.org/wiki/JavaScript}{JavaScript}/\href{https://en.wikipedia.org/wiki/Node.js}{NodeJS}}

% Frameworks/Libraries
\NewCVTagField{CodeMirrorJs}[CodeMirror.js]{https://codemirror.net/}
\NewCVTagField{datGUI}[dat.GUI]{https://github.com/dataarts/dat.gui/}
\NewCVTagField{Defold}{https://defold.com/}
\NewCVTagField{Django}{https://www.djangoproject.com}
\NewCVTagField{Express}{https://expressjs.com/}
\NewCVTagField{Godot}{https://godotengine.org/}
\NewCVTagField{Gulp}{https://gulpjs.com}
\NewCVTagField{MatterJS}{https://brm.io/matter-js/}
\NewCVTagField{MongoDB}{https://www.mongodb.com/}
\NewCVTagField{MySQL}{https://www.mysql.com/}
\NewCVTagField{NextJS}{https://nextjs.org}
\NewCVTagField{NUnit}{https://nunit.org/}
\NewCVTagField{OCRA}{https://github.com/larsch/ocra/}
\NewCVTagField{PFiveJS}[P5JS]{https://p5js.org/}
\NewCVTagField{Phaser}{https://phaser.io/}
\NewCVTagField{PostCSS}{https://postcss.org/}
\NewCVTagField{PostgreSQL}{https://www.postgresql.org}
\NewCVTagField{React}{https://reactjs.org/}
\NewCVTagField{ReactNative}[React Native]{https://reactnative.dev/}
\NewCVTagField{Rollup}{https://www.rollupjs.org/}
\NewCVTagField{SCSSSass}[SCSS/Sass]{https://sass-lang.com/}
\NewCVTagField{Snowpack}{https://www.snowpack.dev/}
\NewCVTagField{SolidJS}{https://www.solidjs.com/}
\NewCVTagField{Stylus}{https://stylus-lang.com/}
\NewCVTagField{UE}[UE4]{https://www.unrealengine.com/en-US/}
\NewCVTagField{Unity}{https://unity.com/}
\NewCVTagField{Webpack}{https://webpack.js.org/}

\begin{document}
	{\setlength{\forkmeoffset}{1in} \forkme[east]{https://github.com/jaacko-torus/resume/}}
	\name{Julián Antonio Avar Campopiano}
	\tagline{Software Developer}
	%% You can add multiple photos on the left or right
	% \photoR{2.8cm}{Globe_High}
	% \photoL{2.5cm}{Yacht_High,Suitcase_High}

	\personalinfo{%
		% Not all of these are required!
		\email{\uline{julian.a.avar@gmail.com}}{mailto:julian.a.avar@gmail.com}
		\phone{\uline{+1\,(773)\,680-0681}}{tel:17736800681}
		\location{\uline{Champaign--Urbana, IL}}{https://www.google.com/maps/place/Champaign\%E2\%80\%93Urbana,+IL}
		\\\smallskip
		% \homepage{www.homepage.com}
		% \twitter{@twitterhandle}
		\linkedin{\uline{julian-avar-campopiano}}{https://linkedin.com/in/julian-avar-campopiano}
		\github{\uline{jaacko-torus}}{https://github.com/jaacko-torus}
		\npm{\uline{jaacko-torus}}{https://www.npmjs.com/~jaacko-torus}
		\itchio{\uline{jaacko-torus}}{https://jaacko-torus.itch.io}

		%% You can add your own arbitrary detail with
		%% \printinfo{symbol}{detail}[optional hyperlink prefix]
		% \printinfo{\faPaw}{Hey ho!}[https://example.com/]
		%% Or you can declare your own field with
		%% \NewInfoFiled{fieldname}{symbol}[optional hyperlink prefix] and use it:
		% \NewInfoField{gitlab}{\faGitlab}[https://gitlab.com/]
		% \gitlab{your_id}
		%%
		%% For services and platforms like Mastodon where there isn't a
		%% straightforward relation between the user ID/nickname and the hyperlink,
		%% you can use \printinfo directly e.g.
		% \printinfo{\faMastodon}{@username@instace}[https://instance.url/@username]
		%% But if you absolutely want to create new dedicated info fields for
		%% such platforms, then use \NewInfoField* with a star:
		% \NewInfoField*{mastodon}{\faMastodon}
		%% then you can use \mastodon, with TWO arguments where the 2nd argument is
		%% the full hyperlink.
		% \mastodon{@username@instance}{https://instance.url/@username}
	}

	\vspace{-0.5cm}
	\makecvheader
	%% Depending on your tastes, you may want to make fonts of itemize environments slightly smaller
	\AtBeginEnvironment{itemize}{\small}
	\newcommand{\hr}{\enspace$|$\enspace}

	%% Set the left/right column width ratio to 6:4.
	\columnratio{0.5}

	% Start a 2-column paracol. Both the left and right columns will automatically
	% break across pages if things get too long.
	\begin{paracol}{2}
		\cvsection{Education}

		\cvevent{Full Sail University}{Bachelors in Computer Science}{October, 2020--Present}{Winter Park, FL}

		\divider

		\cvevent{Bellecour École}{Bachelors in Game Design}{September, 2019--June, 2020}{Lyon, France}

		\cvsection{Experience}

		\cvevent{Software Engineer}{\href{https://www.linkedin.com/company/frasca-international-inc-/}{\uline{Frasca International, Inc.}}}{April 2022--Present}{Champaign--Urbana, IL}
		\begin{itemize}
			\item Creating aircraft simulators to train pilots, using C++ and C\#
		\end{itemize}

		\divider

		\cvevent{Interactive \& Web Developer}{\href{https://www.linkedin.com/company/john-deere/}{\uline{John Deere}}}{January--March 2022}{Davenport, IA}
		\begin{itemize}
			\item Refactored chunks of the JS DevOps code base
			\item Helped creating/maintaining docs for the JS DevOps repositories
			\item Created interactives using Unity
			\item Built library infrastructure for future interactives in C\#
		\end{itemize}

		%%%% Panda Express
		%% \cvevent{Kitchen Staff}{\href{https://www.linkedin.com/company/panda-restaurant-group/}{\uline{Panda Express}}}{October, 2021-November, 2021}{Champaign, IL}
		%%%%

		\divider

		\cvevent{Programming Tutor}{\href{https://www.linkedin.com/company/id-tech-camps/}{\uline{iD Tech}}}{May, 2021--December, 2021}{Campbell, CA}
		\begin{itemize}
			\item Taught general JavaScript, Python, Java, and C\#
			\item Taught 100+ students
			\item Taught age ranges 11--24, $\sim$20\% pursing a college degree
		\end{itemize}

		%%%% Iron Sun Holdings
		%% \cvevent{Intern}{\href{https://www.linkedin.com/company/hayden-organic}{\uline{Hayden Organic/Iron Sun Holdings}}}{May--September, 2021}{NYC, NY}
		%%%%

		\divider

		\cvevent{Game Jam Lead \& Programmer}{\href{https://jimmys-test-site.itch.io}{\uline{Jimmy's Test Site}}}{March, 2020--November, 2021}{Online Group}
		\vspace{-0.3cm}
		\begin{justify}
			Online group created by Abdullah Salem (\href{https://github.com/Fenguinn}{\uline{@Fenguinn}}) and me, later evolved to 10 programmers + 3 artists which took turns aiding our projects. We had people from USA, Brazil, France, and China.
		\end{justify}
		\vspace{-0.2cm}
		\medskip

		\begin{itemize}
			\item Led 10+ people
			\item Taught C\#, GDScript, and git workflows to new collaborators
			\item Took responsability to finish thing where others couldn't
			\item Learned to coordinate development and art teams
		\end{itemize}

		\cvsection{Languages}
		\cvtag{English}
		\cvtag{Spanish}

		% No skill level needed since they are both equal
		% \cvskill{English}{5}
		% \cvskill{Spanish}{5}

		% use ONLY \newpage if you want to force a page break for ONLY the current column
		\newpage

		% Switch to the right column. This will now automatically move to the second page if the content is too long.
		\switchcolumn

		\cvsection{Programming Languages}
		\CVTagC
		\CVTagCPlusPlus
		\CVTagCSharp
		\CVTagGDScript
		% \CVTagGo % Not enough real world experience yet
		\CVTagHTMLCSS
		\CVTagJava \\ % overflow
		\CVTagJSNode
		\CVTagLaTeX
		\CVTagLua
		\CVTagPython
		\CVTagRuby
		\CVTagScala
		\CVTagSQL
		\CVTagTypeScript

		\cvsection{Frameworks/Libraries}
		% .net
		\CVTagNUnit
		% databases
		\CVTagMongoDB
		\CVTagMySQL
		% python
		\CVTagDjango
		% ruby
		\CVTagOCRA \\ % overflow
		% web
		\CVTagCodeMirrorJs
		\CVTagdatGUI
		\CVTagExpress
		\CVTagGulp
		\CVTagNextJS
		\CVTagPostCSS
		\CVTagReact
		\CVTagReactNative
		\CVTagRollup
		\CVTagSCSSSass
		\CVTagSnowpack
		\CVTagSolidJS
		\CVTagStylus
		\CVTagWebpack
		\\ % line break
		% game engines
		\CVTagDefold
		\CVTagGodot
		\CVTagPhaser
		\CVTagUE
		\CVTagUnity

		\cvsection{Other Skills}
		\cvtag{Chrome Extension}
		\cvtag{Git}
		\cvtag{Heroku}
		\cvtag*{Hoppscotch}[https://hoppscotch.io]
		\cvtag{NPM}
		\cvtag*{Postman}[https://www.postman.com]
		\cvtag{Regex}
		\cvtag{Unix}
		\cvtag{WSL}
		\cvtag{Yarn}

		\cvsection{Personal Projects}

		%%%%
		%% Manual line breaks are necesary, since the links seem to be breaking the frames for some
		%% reason *sigh*
		%%%%

		\cvevent
		{\href{https://github.com/jaacko-torus/kodikos/}{\uline{Kodikos}}{\hr}Chrome Extension}
		{{\CVTagHTMLCSS}{\CVTagJavaScript}\cvtag{Chrome Extension}}
		{19--23 November, 2021}{}
		\vspace{-0.3cm}
		\begin{justify}
			Password generator built as a chrome extension. The options menu contains a mini code editor (built with \href{https://codemirror.net/}{\uline{CodeMirror.js}}) to more flexibly configure how passwords are generated.
		\end{justify}
		\vspace{-0.2cm}
		\smallskip
		\begin{itemize}
			\item Used the \texttt{Fetch API} to gather words for password generation
			\item Using an \href{https://developer.mozilla.org/en-US/docs/Web/HTTP/Headers/Content-Security-Policy/sandbox/}{\uline{\texttt{<iframe>} sandbox}} to run \texttt{eval} in a secure context
		\end{itemize}

		\divider

		\cvevent
		{\href{https://github.com/jaacko-torus/lao-mm/}{\uline{LAO\&MM}}{\hr}CLI}
		{{\CVTagRuby}{\CVTagOCRA}}
		{July, 2020}{}
		\vspace{-0.3cm}
		\begin{justify}
			% Wikipedia link cannot have a `/` at the end.
			At tiny CLI that translates \href{https://en.wikipedia.org/wiki/Light_novel}{\uline{Light Novels}} (serialized Japanese novellas) from raw image scan. It does so with \href{http://capture2text.sourceforge.net/}{\uline{OCR}} and cloud translation software.
		\end{justify}
		\vspace{-0.2cm}
		\smallskip
		\begin{itemize}
			\item First useful CLI application
		\end{itemize}

		\divider

		\cvevent
		{\href{https://github.com/jaacko-torus/lox}{\uline{Lox}}{\hr}Programming Language}
		{{\CVTagJava}{\CVTagC}}
		{October 2021--Present}{}
		\vspace{-0.3cm}
		\begin{justify}
			A language interpreter for ``lox'', a toy-language spec.-ed by \href{https://github.com/munificent}{\uline{Bob Nystrom}}, creator of \href{https://en.wikipedia.org/wiki/Dart_(programming_language)}{\uline{Dart}}.
		\end{justify}
		\vspace{-0.2cm}
		\smallskip
		\begin{itemize}
			\item Built initial version in Java using an AST interpreter
			\item In the process of transforming into a C VM bytecode compiler for better performance
		\end{itemize}

		%%%% Space Time Explorer
		%% \cvevent
		%% {\href{https://github.com/jaacko-torus/STE}{\uline{Space-Time Explorer}}{\hr}Game}
		%% {{\CVTagTypeScript}{\CVTagSnowpack}{\CVTagPFiveJS}{\CVTagMatterJS}{\CVTagdatGUI}}
		%% {2015--Present}{}
		%% \vspace{-0.3cm}
		%% \begin{justify}
		%% 	Covered in stains of merge conflicts, this project taught me the importance of git, orderly code, and functional programming concepts.
		%% \end{justify}
		%% \vspace{-0.2cm}
		%% \smallskip
		%% \begin{itemize}
		%% 	\item Initially built directly with web APIs
		%% 	\item Increasing complexity in physics and rendering led me to rely on libraries, instead of making everything myself
		%% 	\item Taught me to be orderly and document my code
		%% 	\item Made a middleware interface
		%% \end{itemize}
		%%%%

		%%%% Sky Fishing
		%% \cvevent
		%% {\href{https://jimmys-test-site.itch.io/skyfishing}{\uline{Sky Fishing}}{\hr}Game}
		%% {\cvtag{Godot}\cvtag{GDScript}}
		%% {22--31 January, 2021}{}
		%% \vspace{-0.3cm}
		%% \begin{justify}
		%% 	Entry to the Feel Good Game Jam.
		%% \end{justify}
		%% \vspace{-0.2cm}
		%% \smallskip
		%% \begin{itemize}
		%% 	\item Learned the importance of concept art/models
		%% 	\item Implemented an inventory
		%% 	\item Used data objects instead of game objects to tie items to their corresponding enemy, increasing performace by a tenfold
		%% \end{itemize}
		%%%%
	\end{paracol}
\end{document}
